
\subsection{Observaciones y conclusiones}

Los resultados obtenidos son razonables, al menos en forma relativa entre los casos analizados, la exactitud del \textbf{THD} calculado en cada caso, es dependiente del programa, y muy dependiente de los modelos específicos usados para los transistores, se intento usar para estos, los modelos proveídos por los fabricantes de cada dispositivo. Como ya se menciono antes, hay una cuestión con la definición de lo que es clase \textbf{A} en una etapa de salida del tipo push-pull, pero al margen de eso, el resultado esperado de que la distorsión sea menor para clase \textbf{A}, que para clase \textbf{B}, se cumple. \\
A pesar de tratarse de un circuito incompleto, faltando los circuitos completos de polarización, las protecciones, etc. Las simulaciones demuestran claramente que al menos la topología es un muy buen punto de partida para un amplificador de audio de muy buena calidad. En particular con el uso de mejores transistores para la etapa de salida, por ejemplo los transistores \textbf{TOSHIBA} que se mencionaron en clase, como algunas otras mejoras que se pueden introducir, como ser una carga activa para el par diferencial con mejor factor de copia y mayor resistencia de salida, o una etapa pre-driver a la salida, que presente al \textbf{VAS} una todavía mayor resistencia, un \textbf{VAS} cascode, etc. Son todas cosas que se pueden incorporar para mejorar aún mas la calidad del amplificador.
