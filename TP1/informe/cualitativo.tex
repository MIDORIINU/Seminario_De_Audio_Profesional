
\normalfont

La topología del circuito corresponde a la de un típico amplificador de potencia de tres etapas realimentado, la tensión de salida es muestreada y sumada a la entrada, formando un lazo de realimentación \textbf{serie-paralelo}, estabilizando la tensión de salida. \\

En el circuito se pueden diferenciar claramente las tres etapas, las mismas son:


\begin{itemize}
\item Amplificador diferencial con caga activa (espejo de corriente con degeneración de emisor): realiza la suma (resta) de la señal de entrada con la señal realimentada y provee amplificación.
\item VAS: Formado por un seguidor polarizado con la tensión de colector de la primera etapa, que provee adaptación de impedancia entre la primera etapa (permitiendo que el par diferencial provea alta ganancia, aprovechando su carga activa) y un emisor común con carga activa (fuente de corriente simple con degeneración de emisor), que provee la ganancia. Es en esta etapa que se realiza la compensación de Miller con un capacitor.
\item Etapa de salida push-pull formada con Darlingtons complementarios.
\end{itemize}

Con un muy rápido análisis aproximado por inspección, se obtiene una ganancia para el amplificador de tres etapas de alrededor de $130 \si[per-mode=symbol]{\decibel}$, obteniéndose con este valor para la ganancia de lazo, alrededor de $110 \si[per-mode=symbol]{\decibel}$, este valor se ajusta muy bien para un análisis aproximado, con el valor obtenido por simulación a frecuencias medias.






