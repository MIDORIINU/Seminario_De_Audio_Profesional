
El capacitor $C_{1}$ que se coloca para impedir que continua aplicada al circuito por la fuente de la señal de audio, o una etapa previa, altere la polarización, debería ser un capacitor electrolítico no polarizado, y su valor de $100 \si[per-mode=symbol]{\micro\farad}$ se seleccionó para cumplir con una frecuencia de corte inferior de como máximo $20 \si[per-mode=symbol]{\hertz}$, y no influir significativamente en el THD a bajas frecuencias. Con un valor de $47 \si[per-mode=symbol]{\micro\farad}$ sería suficiente para cumplir con la frecuencia de corte inferior, pero su influencia sobre el THD se hace apreciable incluso a $1 \si[per-mode=symbol]{\kilo\hertz}$ \\

El capacitor $C_{2}$ de $1000 \si[per-mode=symbol]{\micro\farad}$ se seleccionó para que no tenga efecto apreciable sobre el THD, especialmente a bajas frecuencias, y debería ser un capacitor electrolítico de buena calidad, de bajo ESR y no polarizado. La calidad de este capacitor es crítica por hallarse en la red de realimentación, influyendo directamente en la calidad y estabilidad del amplificador. En un circuito mas completo deberían agregarse diodos que protejan este capacitor de un valor de tensión que podría aparecer a la salida, en caso de saturación, que supere su máxima tensión de trabajo. \\

El capacitor $C_{C}$, de $39 \si[per-mode=symbol]{\pico\farad}$, ver sección~\sectref{sect:compensation}, que se coloca para realizar la compensación de Miller, debería ser un capacitor de poliestireno o poliéster.


\vfill



