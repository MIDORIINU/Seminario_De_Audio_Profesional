
\subsection{Punto 1}

\textbf{Enunciado}: \textsl{Analizar que función cumple y como opera el subcircuito compuesto por $R_{12}$ a $R_{17}$, $C_{16}$ y $U_{3}A$. Luego incluir $R_{S}$. ¿Qué características tiene éste subcircuito, por ejemplo, su transferencia, su ancho de banda, su dependencia de las especificaciones del amplificador operacional $TL082$, de sus fuentes de alimentación, de la temperatura, de la tolerancia y tecnología de los resistores con los que se lo implemente, etc.}


\vspace{1.5cm}


Se explica en detalle en \sectref{section:modelo_operacional}.



\subsection{Punto 2}

\textbf{Enunciado}: \textsl{Analizar qué función cumple y como opera el subcircuito compuesto por $R_{18}$ a $R_{19}$, $C_{15}$ y $U_{3}B$. ¿Qué características tiene éste subcircuito, por ejemplo, su transferencia, su ancho de banda, su dependencia de las especificaciones del amplificador operacional $TL082$, de sus fuentes de alimentación, de la temperatura, de la tolerancia y tecnología de los resistores con lo que se lo implemente, etc. ($R_{18}$ puede variarse desde $0 \si[per-mode=symbol]{\ohm}$ a $18 \si[per-mode=symbol]{\kilo\ohm}$).}

\vspace{1.5cm}


Se explica en detalle en \sectref{section:modelo_operacional}.


\subsection{Punto 3}

\textbf{Enunciado}: \textsl{Analizar qué función cumple y cómo opera el subcircuito compuesto por $R_{20}$ a $R_{23}$ y $Q_{7}$-$Q_{9}$-$Q_{10}$-$Q_{11}$. ¿Qué características tiene éste subcircuito?.}


\vspace{1.5cm}

El circuito se explica en forma cualitativa y con simulaciones en la sección~\sectref{section:switch}. A partir de la sección~\sectref{section:switch_small_signal_begin} y hasta la sección sección~\sectref{section:switch_small_signal_end}, se analiza la transferencia en pequeña señal como parte de las redes de realimentación del circuito.


